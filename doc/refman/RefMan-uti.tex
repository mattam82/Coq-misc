\chapter[Utilities]{Utilities\label{Utilities}}

The distribution provides utilities to simplify some tedious works
beside proof development, tactics writing or documentation.

\section[Building a toplevel extended with user tactics]{Building a toplevel extended with user tactics\label{Coqmktop}\index{Coqmktop@{\tt coqmktop}}}

The native-code version of \Coq\ cannot dynamically load user tactics
using Objective Caml code. It is possible to build a toplevel of \Coq,
with Objective Caml code statically linked, with the tool {\tt
  coqmktop}.

For example, one can build a native-code \Coq\ toplevel extended with a tactic
which source is in {\tt tactic.ml} with the command 
\begin{verbatim}
     % coqmktop -opt -o mytop.out tactic.cmx
\end{verbatim}
where {\tt tactic.ml} has been compiled with the native-code
compiler {\tt ocamlopt}. This command generates an executable
called {\tt mytop.out}. To use this executable to compile your \Coq\
files, use {\tt coqc -image mytop.out}.

A basic example is the native-code version of \Coq\ ({\tt coqtop.opt}),
which can be generated by {\tt coqmktop -opt -o coqopt.opt}.


\paragraph[Application: how to use the Objective Caml debugger with Coq.]{Application: how to use the Objective Caml debugger with Coq.\index{Debugger}}

One useful application of \texttt{coqmktop} is to build a \Coq\ toplevel in
order to debug your tactics with the Objective Caml debugger.
You need to have configured and compiled \Coq\ for debugging
(see the file \texttt{INSTALL} included in the distribution).
Then, you must compile the Caml modules of your tactic with the
option \texttt{-g} (with the bytecode compiler) and build a stand-alone
bytecode toplevel with the following command:

\begin{quotation}
\texttt{\% coqmktop -g -o coq-debug}~\emph{<your \texttt{.cmo} files>}
\end{quotation}


To launch the \ocaml\ debugger with the image you need to execute it in
an environment which correctly sets the \texttt{COQLIB} variable.
Moreover, you have to indicate the directories in which
\texttt{ocamldebug} should search for Caml modules.

A possible solution is to use a wrapper around \texttt{ocamldebug}
which detects the executables containing the word \texttt{coq}. In
this case, the debugger is called with the required additional
arguments. In other cases, the debugger is simply called without additional
arguments. Such a wrapper can be found in the \texttt{dev/}
subdirectory of the sources. 

\section[Modules dependencies]{Modules dependencies\label{Dependencies}\index{Dependencies}
  \index{Coqdep@{\tt coqdep}}}

In order to compute modules dependencies (so to use {\tt make}),
\Coq\ comes with an appropriate tool, {\tt coqdep}.

{\tt coqdep} computes inter-module dependencies for \Coq\ and
\ocaml\ programs, and prints the dependencies on the standard
output in a format readable by make.  When a directory is given as
argument, it is recursively looked at.

Dependencies of \Coq\ modules are computed by looking at {\tt Require}
commands ({\tt Require}, {\tt Requi\-re Export}, {\tt Require Import},
but also at the command {\tt Declare ML Module}.

Dependencies of \ocaml\ modules are computed by looking at
\verb!open! commands and the dot notation {\em module.value}. However,
this is done approximatively and you are advised to use {\tt ocamldep}
instead for the \ocaml\ modules dependencies.

See the man page of {\tt coqdep} for more details and options.


\section[Creating a {\tt Makefile} for \Coq\ modules]{Creating a {\tt Makefile} for \Coq\ modules\label{Makefile}
\index{Makefile@{\tt Makefile}}
\index{CoqMakefile@{\tt coq\_Makefile}}}

When a proof development becomes large and is split into several files,
it becomes crucial to use a tool like {\tt make} to compile \Coq\
modules.

The writing of a generic and complete {\tt Makefile} may be a tedious work
and that's why \Coq\ provides a tool to automate its creation,
{\tt coq\_makefile}.

Arguments are explain by \texttt{\% coq\_makefile --help}.  They can be directly
written in the command line but it is recommended to write them in a file (called
for example {\tt Make}) and then call {\tt coq\_makefile -f Make -o
  Makefile}. That means options are in {\tt Make} file and output is {\tt
  Makefile} This way, {\tt Makefile} will be automatically regenerated if
something changes in {\tt Make}.

The first time you use this tool, you may be happy with:
\begin{quotation}
\texttt{\% \{ echo '-R .} {\em MyFancyLib} \texttt{' ; find -name '*.v' -print \} >
  Make \&\& coq\_makefile -f Make -o Makefile}
\end{quotation}

To customize things afterwards, remember:
\begin{itemize}
\item Coq files must end in {\tt .v}, caml modules in {\tt .ml4} if they
  require camlp preproccessing (and in {\tt .ml} otherwise), and caml module signatures in {\tt
    .mli}.
\item If you give a directory directly as argument, it is because you provide a
  Makefile for it in it.
\item {\tt -R} option is for Coq, {\tt -I} for caml. The same directory can
  ``included'' by both.
  Using {\tt -R} option gives a right logical path and a correct installation
  emplacement to your coq files.
\item If your files depend on an external library that isn't install somewhere
  looked by coqc, use {\tt OTHERFLAGS = '-R path/to/lib lib\_name'} option in your {\tt
    Make} but don't do {\tt -R \dots} directly, the {\em make clean} command would
  erase it!
\end{itemize}

\Warning To compile a project containing \ocaml{} files you must keep
the sources of \Coq{} somewhere and have an environment variable named
\texttt{COQTOP} that points to that directory.

\section[Documenting \Coq\ files with coqdoc]{Documenting \Coq\ files with coqdoc\label{coqdoc}
\index{Coqdoc@{\sf coqdoc}}}

\input{./coqdoc}

\section{Exporting \Coq\ theories to XML}

\input{./Helm}

\section[Embedded \Coq\ phrases inside \LaTeX\ documents]{Embedded \Coq\ phrases inside \LaTeX\ documents\label{Latex}
  \index{Coqtex@{\tt coq-tex}}\index{Latex@{\LaTeX}}}

When writing a documentation about a proof development, one may want
to insert \Coq\ phrases inside a \LaTeX\ document, possibly together with
the corresponding answers of the system. We provide a
mechanical way to process such \Coq\ phrases embedded in \LaTeX\ files: the
{\tt coq-tex} filter.  This filter extracts Coq phrases embedded in
LaTeX files, evaluates them, and insert the outcome of the evaluation
after each phrase.

Starting with a file {\em file}{\tt.tex} containing \Coq\ phrases,
the {\tt coq-tex} filter produces a file named {\em file}{\tt.v.tex} with
the \Coq\ outcome. 

There are options to produce the \Coq\ parts in smaller font, italic,
between horizontal rules, etc.
See the man page of {\tt coq-tex} for more details.

\medskip\noindent {\bf Remark.} This Reference Manual and the Tutorial
have been completely produced with {\tt coq-tex}.


\section[\Coq\ and \emacs]{\Coq\ and \emacs\label{Emacs}\index{Emacs}}

\subsection{The \Coq\ Emacs mode}

\Coq\ comes with a Major mode for \emacs, {\tt coq.el}. This mode provides
syntax highlighting (assuming your \emacs\ library provides
{\tt hilit19.el}) and also a rudimentary indentation facility
in the style of the Caml \emacs\ mode.

Add the following lines to your \verb!.emacs! file:

\begin{verbatim}
  (setq auto-mode-alist (cons '("\\.v$" . coq-mode) auto-mode-alist))
  (autoload 'coq-mode "coq" "Major mode for editing Coq vernacular." t)
\end{verbatim}

The \Coq\ major mode is triggered by visiting a file with extension {\tt .v},
or manually with the command \verb!M-x coq-mode!.
It gives you the correct syntax table for
the \Coq\ language, and also a rudimentary indentation facility:
\begin{itemize}
  \item pressing {\sc Tab} at the beginning of a line indents the line like
    the line above;

  \item extra {\sc Tab}s increase the indentation level
    (by 2 spaces by default);

  \item M-{\sc Tab} decreases the indentation level.
\end{itemize}

An inferior mode to run \Coq\ under Emacs, by Marco Maggesi, is also
included in the distribution, in file \texttt{coq-inferior.el}.
Instructions to use it are contained in this file.

\subsection[Proof General]{Proof General\index{Proof General}}

Proof General is a generic interface for proof assistants based on
Emacs (or XEmacs). The main idea is that the \Coq\ commands you are
editing are sent to a \Coq\ toplevel running behind Emacs and the
answers of the system automatically inserted into other Emacs buffers. 
Thus you don't need to copy-paste the \Coq\ material from your files
to the \Coq\ toplevel or conversely from the \Coq\ toplevel to some
files. 

Proof General is developped and distributed independently of the
system \Coq. It is freely available at \verb!proofgeneral.inf.ed.ac.uk!.


\section[Module specification]{Module specification\label{gallina}\index{Gallina@{\tt gallina}}}

Given a \Coq\ vernacular file, the {\tt gallina} filter extracts its
specification (inductive types declarations, definitions, type of
lemmas and theorems), removing the proofs parts of the file. The \Coq\
file {\em file}{\tt.v} gives birth to the specification file
{\em file}{\tt.g} (where the suffix {\tt.g} stands for \gallina).

See the man page of {\tt gallina} for more details and options.


\section[Man pages]{Man pages\label{ManPages}\index{Man pages}}

There are man pages for the commands {\tt coqdep}, {\tt gallina} and
{\tt coq-tex}. Man pages are installed at installation time
(see installation instructions in file {\tt INSTALL}, step 6).

%BEGIN LATEX
\RefManCutCommand{ENDREFMAN=\thepage}
%END LATEX

%%% Local Variables: 
%%% mode: latex
%%% TeX-master: t
%%% End: 
